\documentclass[10pt,a4paper]{report}
\usepackage[T1]{fontenc}
\usepackage[utf8]{inputenc}
\usepackage{amsmath}
\usepackage{amsfonts}
\usepackage{amssymb}
\usepackage{graphicx}
\usepackage{tikz, pgfplots}
\usepackage{filecontents}
\usepackage{verbatim}
\usepgfplotslibrary{fillbetween}
\usepackage{url}


\author{D. Doimo}

\begin{document}
\textit{Set $g(x)=(1-e^{-1})x^{1/2}$, $f(x)=1-e^{-x}$ show that $f(x) \leq g(x) \quad \forall x\in[0,1]$.\vspace{5mm}}
$Dim$\\
$f,g \subset C^{\infty}(0,1)$ and $f(0)=g(0)=0$,  $f(1)=g(1)=1-e^{-1}$; if $f(x) \leq g(x) \quad \forall x\in[0,1]$, these points must be the only one in which the two functions are equal in $(0,1)$. \\
For the Cauchy's mean value theorem, for any ${a,b}$ for which$f(a)=g(a)$ and $f(b)=g(b)$ must exist at least a $c\in(a,b)$ that realizes $f'(c)=g'(c)$. We will now proove that in $(0,1)$ there is only one $c$ that realizes the equality of the first derivatives and therefore other points for which $f=g$ can't exist.\\
$f'(x)=g'(x)$ gives:
\begin{equation}
e^{-x}=\dfrac{1-e^{-1}}{2x^{1/2}}\quad \Rightarrow \quad x^{1/2}e^{-x}=\dfrac{1-e^{-1}}{2}\doteq h(x)
\end{equation}
Now the following inequalities:
\begin{equation}
h(0)=0<(1-e^{-1})/2, \quad h(1)=e^{-1}>(1-e^{-1})/2
\end{equation}
(the second being equivalent to $(3>e)$) imply, for the continuity of $h$, the existence of at least one point of intersection between $h$ and $(1-e^{-1})/2$.\\
We will show at the end of the proof that $h$ is concave in $(0,1)$. Given this let's choose the first $x^{*}$ for which $h(x)=(1-e^{-1})/2$. \\
Then by definition of concavity we have:

\begin{equation}
h(x^{*}+(1-x^{*})t)> h(x^{*})+(h(1)-h(x^{*}))t \quad \forall t \in [0,1]
\end{equation}
but we have:

\begin{equation}
h(x^{*})=\dfrac{1-e^{-1}}{2}, \quad h(1)-h(x^{*})>0
\end{equation}
therefore

\begin{equation}
h(x)>\dfrac{1-e^{-1}}{2} \ \forall x \in(x^{*},1]
\end{equation}
and we conclude that there exist only one point $c\equiv x^{*}$ in $(0,1)$ for which $f'(c)=g'(c)$.\\

Finally $h$ is indeed concave since:

\begin{equation}
h''(x)=\dfrac{e^{-x}}{x^{1/2}}\left(-1+x-\dfrac{1}{4x}\right)<0 \ \forall x\in[0,1]
\end{equation} 
\\
\textit{Show how to sample from the distribution $p_{X}(x)=3/2\sqrt{x}$.}\\

Since the pdf is a power law it is esay to use the cumulative method to generate random numbers from it. The cumulative distribution function is:
\begin{equation}
F(x)=x^{3/2} \quad x \in [0,1]
\end{equation}
The prescription is then to take the inverse that will allow to pass from a unifom distribution to the $ p_{X}(x)$ we seek. More explicitly let $y_{i}$ be a random number from a uniform distribution in the interval $[0,1]$ then:

\begin{equation}
x_{i}=F^{-1}(y_{i})=y_{i}^{2/3}
\end{equation}
will be distributed according to $p_{X}(x)=3/2\sqrt{x}.$
\\
\textit{Show how to compute the integral I.}\\

The integrand is an average weighted according to a Gaussian distribution. Since the joint pdf is the product of three standard gaussian distributions, to compute the integral we first generate $x_{i}, y_{i}, z_{i}$ each one from a standard gaussian distribution, then we compute the estimate with the following formula:
\begin{equation}
I=\dfrac{\sum_{i}^{N }g(x_{i}, y_{i}, z_{i})}{N}, \quad g(x,y,z)=\vert \cos{\sqrt{x^{2}+y^{2}}} \vert \tanh{(x^{2}+y^{2}+z^{4})}
\end{equation} 
\end{document}

